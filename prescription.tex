\section{prescription}

	\paragraph{Introduction}

It would not make for reasonable or practical regulation to force all application developers to adopt strong privacy measures as many firms make a significant amount of revenue from these sources, likewise, it would not suffice to do nothing and let the current situation stand as it is. Some users, needless to say, would not mind giving up their personal data for the tradoff of using an application, in these situations the current scheme works fine, what is not acceptable however, are privacy concerned users who wish to use an application's serivce and are dubiously coaxed into agreeing to that applications information practices which do not align with his/her expectations. Furthermore, the demand placed on privacy concerned applications by these users is not a strong enough market force to incite change in the industry. Policy must step in to shift this sub-optimal norm to one that satisfies all parties. 



One of the reasons it is difficult to form good regulation around the issue of personal data collection is because the situation is not a binary good vs evil system. There are many forms of personal data collection that are required for a service to run correctly and for keeping state of a user through a workflow. 



    % Thankfully, Apple has announced that it will be deprecating the use of the UDID in IOS6 due to the privacy concerns. MORE INFO NEEDED HERE

\paragraph{FTC should push vendors to regulate apps}


% NOTE: There is already a norm in place to pay for apps 

% NOTE: Ad networks should want more transparency as to avoid backlashes which could lead to potential regulations that hinder their business practices. 
% Use path example => Uploading contact list information and subsequent blacklash from community 


	\subsection{Dangers of Explicit Statutes} 

One danger that must be mentioned when addressing any regulation regarding the gathering of smartphone data is the possible societal impacts of limiting the use of such data. Certainly, not all data mining is evil. Many important discoveries and innovations would not have been possible withouth the unrestricted access to personal data. 

In one case, researchers at Harvard School of Public Health carried out a study in Kenya that consisted of analysing the location data of nearly 15 million mobile phones to try and understand the infection patterns of malaria\cite{unlock}. The study found that human travel carried the infection more so than travel by mosquitoes, and that the spread of the infection followed stable patterns every year. Ultimately, the study identified areas where malaria treatment centers would be most effective, and infection rates have gone down \%25 since 2000 due targeted prevention. 

Certainly, some uses of mobile data can benifit an entire population, and as such, any policy should not block the legitimate uses of personal, mobile data when the user consents to its use. Of course, certain measures should be in place to ensure that the individuals consent to the use of that data, and that when it is collected, it is only used for the intended purposes, and is secure such that no leakage occurs that could harm the individual. 