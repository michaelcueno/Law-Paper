\section{prescription}

	\paragraph{Introduction}

It would not make for reasonable or practical regulation to force all application developers to adopt strong privacy measures as many firms make a significant amount of revenue from these sources, likewise, it would not suffice to do nothing and let the current situation stand as it is. Some users, needless to say, would not mind giving up their personal data for the trade-off of using an application, in these situations the current scheme works fine, what is not acceptable however, are privacy concerned users who wish to use an application's service and are dubiously coaxed into agreeing to that applications information practices which do not align with his/her expectations. Furthermore, the demand placed on privacy concerned applications by these users is not a strong enough market force to incite change in the industry. Policy must step in to shift this sub-optimal norm to one that satisfies all parties. 


Because constructing any sweeping regulation in the mobile data space would greatly hinder technological advancement\cite{???}, and certainly, legislation should never be tied down to specific technologies, 


One of the reasons it is difficult to form good regulation around the issue of personal data collection is because the situation is not a binary good vs evil system. There are many forms of personal data collection that are required for a service to run correctly and for keeping state of a user through a workflow. 



    % Thankfully, Apple has announced that it will be deprecating the use of the UDID in IOS6 due to the privacy concerns. MORE INFO NEEDED HERE

\paragraph{FTC should push vendors to regulate apps}


% NOTE: There is already a norm in place to pay for apps 



	\subsection{Dangers of Explicit Statutes} 

One danger that must be mentioned when addressing any regulation regarding the gathering of smartphone data is the possible societal impacts of limiting the use of such data. Certainly, not all data mining is evil. Many important discoveries and innovations would not have been possible withouth the unrestricted access to personal data. 

In one case, researchers at Harvard School of Public Health carried out a study in Kenya that consisted of analysing the location data of nearly 15 million mobile phones to try and understand the infection patterns of malaria\cite{unlock}. The study found that human travel carried the infection more so than travel by mosquitoes, and that the spread of the infection followed stable patterns every year. Ultimately, the study identified areas where malaria treatment centers would be most effective, and infection rates have gone down \%25 since 2000 due targeted prevention. 

Certainly, some uses of mobile data can benifit an entire population, and as such, any policy should not block the legitimate uses of personal, mobile data when the user consents to its use. Of course, certain measures should be in place to ensure that the individuals consent to the use of that data, and that when it is collected, it is only used for the intended purposes, and is secure such that no leakage occurs that could harm the individual. 


	\subsection{Building on Existing Statutes to Protect Mobile Data}

Fortunately, there are already in place a number of statutes regarding the security of data sets containing personal data. 

%They simply must be extended to the smartphone realm. 

 	\subsection{Technological Solution}
% Talk about how the market can help solve the privacy problem by allowing users to select what applications are allowed to access which data sets
% Like what apple ios is doing right now. 
Any regulation that uses specific, technological requirements in its language would become anachronistic rather quickly as technology advances. Thus, any statute attempting to resolve privacy issues through smartphones specifically should be avoided. Instead, the FTC can use its influence to push recommendations and best practices onto the mobile ecosystem in order to achive an industry regulated market. 
Possibly under threat of creating what is known as a Trade Rule, the FTC could push mobile phone platforms to create better privacy controls for the phones that they run on. 
Since data collection occurs through the operating system of the mobile phone via permissioned access to specific data sets, the operating system is in the best position to block the application of these data. 

Apple's IOS 6 has made great progress in this direction, offering users a privacy control panel. The panel options start with the different types of data that may be collected, location and contacts for example, and upon tapping a specific data type, IOS 6 shows the user all the applications that request access to that data. The user can then, app by app, grant or revoke access to that application's use of the data.

This kind of control is applauded by privacy advocates and is exactly the kind of innovative solutions that only the platform providers can provide. 

The very existence of this kind of control would allow consumers to be more aware of the data collection practices of certian applications. Currently, Android users are only made aware of the types of data that an application will collect when they initially download the application. As mentioned previously, the user typically does not pay attention to these notices nor have a choice other than to be excluded from using the service that the application provides if they disagree with the practices.
With good privacy controls in place however, the mechanism of notice and choice becomes more effective because the user can choose to still use the application, but in a way that retains her privacy preferences.

As the consumer is more easily able to deny applications the data that they request, application will be pushed to make more clear and consice arguments for the use of that data. In this way, users will become more aware of the value that their data holds and a norm will develop around personal, mobile data that makes it more clear to the consumer that that data is valueable. 
This, in turn, would push application developers to make better offers to consumers to user their data. 

 	\subsection{Regulatory Solution}
% This would be more related to the backend of data collection, What uses are appropriate for data collection comapnies,

While this type of technological approach works great to protect consumers against data leakage from specific data sets such as location data, it is inherintly limited by the technology that implements it. Of course any data that the user supplies to the application directly, username, email, date of birth or application useage statistics, cannot be blocked by the operating system. This type of limitation would be too sweeping and make it extreamely difficult for application developers to create useful applications. 
Another area that any technological approach could not influence is the process of what happens to the personal data after it has been collected. These holes in the solution beg resolution through regulation. 

These problems are also good candidates for regulation because any regulation passed in this light would have benifits that carry through more than just the mobile world. Consider privacy regulation on personal data would to to the online tracking practices of the current day. \hl{this is shit}




% Notices should be required, concise and clear for any application requesting access to personal data sets 

% Notices should be pushed to end users in the event of revision. (Users should also consent) 


