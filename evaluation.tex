\section{Evaluation}

	% \subsection{Something Needs To Be Done}

		% \paragraph {Users are weary of tracking}

One of the most troubling aspects to the issue of privacy on mobile devices is the gaping distance between consumers' expectations of privacy and the actual reality of privacy on smartphones. In a survey taken last year, 78 percent of cell phone users think that their personal data such as contact lists, location data, name, address, etc, is at least as private as the data on their home computers\cite{Jenn2012}. This finding has a small caveat in that it assumes the survey respondents consider their home computers to be private, but this hardly seems like a controversial claim. If that is the case, then it would follow that the majority of smartphone users regard their personal information as secure and private. 

% Doesnt go here 
% This argument alone could stand to reason that the notice and choice framework is failing the average consumer based on the fact that they are unaware of the privacy implications in the use of smartphone applications. 

In reality, underneath the hood of the mobile ecosystem, many applications obtain and use personal information on a regular basis. 
A study done by The Wall Street Journal analyzed 101 popular apps and showed that 56 applications sent the phone's unique identifier to third parties without notice or consent\cite{Thru2010}. Many of these applications also attached many other types of personal data to the unique identifiers such as contacts, age, and gender before sending them to various companies. Since then, however, many of the application developers outed by the publication have adopted privacy policies. While it is beneficial to have the privacy practices documented in this form, the very mechanism of notice and choice is failing the average consumer, a point discussed further on. 

		% \paragraph {Growth patterns}

% Despite the fact that consumer awareness is rising with respect to data collection on mobile platforms, the trends of collection and aggregation continue to rise\cite{meeker}. This should not be taken as a suprise as advertising networks have largely been given the go ahead from a legal perspective. The commonly cited case bieng DoubleClick vs Bose determined difinitively that tracking a user's behavior amongst multiple contexts cannot be ciminalized as long as that collection is not tortious\hl{FACT CHECK}. 
% Certianly, the demand placed on targeted advertising via ad networks is not going to go away. Thus, we will continue to see \hl{data collection} companies compete and innovate over ways to more efficiently and more preciscly pinpoint your interests and thus, your suceptibility to interact with an advertisement. 


		% \paragraph {Privacy Dangers}

% One of the more violating practices that is emerging is the use of location data to serve more relevant ads to consumers. 

	\subsection{Current Policy and Legislation}

		% \paragraph{Overview of current protections}
The current legal landscape surrounding the privacy of mobile data collection is a generally passive one except in certain circumstances. The Federal Trade Commission (FTC) is the agency responsible for bringing suits against companies that do not abide by privacy law. Of course their powers only go as far as current statutes allow. There are only a few statutes that can be applied to the issues of smartphone data collection as it occurs in the context of advertising purposes. 

Initially, in an attempt to better understand the industry\cite{Wsj2013} and possibly to induce companies to publish privacy policies, the FTC carried out a criminal investigation on mobile app developers under the pretenses of the Computer Fraud and Abuse Act (CFAA). In particular, the investigation centered around the question of whether the collection of a user's personal information through smartphone applications without the user's notice or consent could be considered computer fraud. Ultimately, the investigation did not lead to any suits, however, it did prompt mobile app developers to take the issue of publishing privacy policies more seriously. 

One case where the FTC had success in bringing actions against mobile data collection, was in a case involving the Children's Online Privacy Protection Act (COPPA). The act requires parental consent before collecting or sharing any information about any child under the age of 13. The lawsuit identified a number of companies that clearly marketed their applications to children and then collected emails and allowed users, mainly children, to post to the Internet in the form of blogs among other things\cite{ftc2011}. The defendants quickly settled to the tune of \$50,000. 


		% \paragraph{Why notice and choice is pursued}

% Explanation of FTC's fair information practices 
For the general case of tracking adults through application use however, the FTC does not seem to want to push its weight around. 
Although the FTC states five principals in regards to fair information practices, notice/awareness, choice/consent, access/participation, integrity/security and enforcement/redress, they have, in practice not pursued them equally with the same rigor.
In part due to a recognition of the importance of the free flow of information in today's economy\cite{privacy-online} they have fallen back, almost entirely, to upholding the principals of notice and choice. 
Also, the commission's concept of notice includes in it common themes of many other fair information privacy principals such as the APEC privacy framework's collection limitation and uses of personal information which limits the collection and use of personal data to only that which is relevant\cite{fair}. 
In other words, the FTC punts on the concepts of fair use and collection by allowing notices to waive them. 
Effectually, the commission's approach allows for contractual like notices, that can contain anything the drafting party desires, no matter how ``unfair'' the practices\cite{fair}. 

% How we got to enforcing only violations of published privacy policies
In practice, the FTC 
has a limited arsenal of weaponry to combat the privacy issues raised on smartphones. The enforcement strategy that they have adopted has been to 
encouraged companies and mobile app developers to publish privacy policies
and then to pursue those entities that violate their policies. 

	\subsection{Failures of notice and choice}

% Basically just sum up Sloan's paper 
The theory of notice and choice for protecting consumer privacy is a wonderful solution to a very real problem. However, when applied to mobile applications, its practical implementation is truly horrendous. Notices are often full of legal jargon and verbose to the point that many consumers do not bother to read them\cite{ftcworkshop}. They also frequently are not even seen by the consumer, instead implicitly agreed upon through using the application. Furthermore, applications are not required to post privacy policies and many applications simply do not\cite{Thru2010}. This sums to a situation where the user of the app is unable to make an informed decision on whether to accept the trade-off of allowing access to personal information with the service of the app.

Even if notices are provided and clear, the implementation of choice is also troubling. The only mechanism of choice that a user is given 
when confronted with an application that does not align with their privacy preferences
is to not use that application. Assuming that the user chooses not to use that application based solely on its privacy policies, there is rarely an alternative to that application that differs only in the way it handles privacy. For instance, if a consumer wishes use Amazon as their mobile, online shopping outlet, but disagrees with it's terms of service or privacy policy, there are not likely to be any other online shopping applications that offer their services without burdening the user to agree to a similar privacy policy\cite{policies}. Thus, the privacy concerned user makes not only the choice to not use the app, but the choice to not use any application that offers a similar service.

% This is the meat of the argument that notice and choice is failing 

% While this paper does not attempt to focus on the issues related to web browser tracking, there is one parallel that is particularly useful from that realm. That is the use of what is called the `Flash' cookie, or `Supercookie'. These are types of cookies that are very difficult to delete because they can re-spawn themselves. 

Ideally, notice and choice would allow consumers to make informed decisions about the services and third parties that they expose their data to. Consumers would be aware of the complex ways that their data would promulgate through the advertising ecosystem and could more readily make an informed decision to consent or deny those practices without being barred from the service in the case of the latter outcome. Unfortunately, the ideal is far from achieved in practice as noted earlier, and we are left with a void of consumer awareness and a lack of any meaningful choice and instead consumers use applications and allow them to collect their personal data through passive acquiescence\cite{sloan}. 


% In this case, some, if not many, consumers would not mind giving up their personal data for the trade-off of using an application. 
% Unfortunately, it is much harder to achieve this ideal in practice, and what has been observed in the United States' implementation of notice and choice 
% What is not acceptable however, are privacy concerned users who wish to use an application and are dubiously coaxed into agreeing to that application's information practices which do not align with her expectations. 


