\section {Introduction}

Privacy issues on smartphone platforms in the United States have been likened to the Wild West due to a lack of direct regulation \cite{Priv2013} and very little consumer awareness \cite{Jenn2012}. As more users leave traditional consumption outlets such as TV, print and even web browser for mobile consumption \cite{Abi2012}, more research and development goes into procedures for tracking behaviors on these platforms.

We have seen a healthy debate for consumer protection in the realms of web browser privacy with the introduction of do not track mechanisms \cite{W3C2012} and default settings which block third party tracking cookies \cite{Moz2013}. The success of these developments can be argued, 
but the presence of a conversation and and beginnings of processes to deal with the privacy issues are encouraging. 
% but the issues seem to be heading in the right direction. 
Unfortunately, it is less clear if the same thing can be said about the privacy issues in the mobile arena. 
Attribution for this can go, in part, to a lack of consumer awareness of the tracking practices that are carried out on mobile devices\cite{Jenn2012}. 
This is increasing alarming as today$'$s smartphones have access to much more personal data sets than do your typical web browser, such as contact lists, text messages and location data to name a few. 

% The attribution for this void is can be placed on a lack of consumer awareness and a cold legal climate surrounding the use and aggregation of personal data acquired through mobile platforms.

The void of consumer awareness is begining to be filled with a growing number of publications that focus on current practices in mobile data collection. The Wall Street Journal, for example, has devoted an entire series to the issue \cite{Wsj2013}. I argue that as consumer awareness increases, so too will the demand for privacy on mobile platforms. This relationship is supported by a recent study\cite{Trust2013}, showing that over the past year, 72 percent of smartphone users are more concerned about privacy and 81 percent choose to avoid applications that they think will not protect their privacy. 

This paper illustrates the privacy implications that arise from the collection of personal, mobile data and suggests a two tiered approach to solving the issue. First, from a tecnhological, industry regulated perspective, and secondly, with some suggestions for leglislation to plug the holes that cannot be filled with the technological approach. 
