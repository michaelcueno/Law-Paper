\section{Background}

	\subsection{Evolution of Behavioral Tracking}
The use of tracking software to model individual behaviors and actions on the internet came into its heyday with the proliferation of the third party cookie. This unique character string lives under the hood of your web browser and can be used for legitimate reasons such as keeping you logged into a session with your bank’s website, but the technology can also be used to collect data about you across many different sites. The use of cookies to perform behavioral tracking has exploded in recent years and shows no signs of slowing down\cite{Thru2010}.  (http://www.law.berkeley.edu/privacycensus.htm)

	\subsection{The migration to smart phones}

As the use of mobile platforms continues to grow, we will continue to see the the same patterns in how media is consumed. A recent survey found that smartphone and tablet users watch, on average, 30 percent less TV and consume even less print media. (Source: ABI) This shift in consumption patterns along with the prospect of advertising to individuals instead of groups have pushed advertisers to invest heavily into mobile tracking technologies for the purposes of behavioral advertising.

   

	\subsection{Technical Discussion of Behavioral Tracking on SmartPhones}

A majority of tracking on mobile platforms is done through the applications that the user downloads through the respective application vendors, most commonly, the Android Market and the App store from Apple. These applications often obtain access to permissioned data such as age, gender, location and the device’s unique identifier. For smartphones running the Android OS, this is typically the International Mobile Equipment Identifier (IMEI) and for Apple products it is the UDID.

Application developers can access this identifier and send application usage statistics along with other permissioned data to advertising networks. When multiple applications send data to the same networks, all using the same identifier, a secret dossier can be established and maintained by that ad network for each device.

A troubling aspect to the use of these unique identifiers is that they cannot be cleared or reset on a device much like cookies on a web browser can. In fact, some countries have made it illegal to change, or spoof, these unique identifiers for anti-theft reasons. (SOURCE) This is in part, what makes smartphone tracking so appealing to advertisers. They are guaranteed a persistent way to identify an individual across multiple contexts.

At the moment the only technical protection that users have against this type tracking procedure is through permission awareness from the application vendors. The Android market for example, informs users what applications will access via a permissions screen prior to downloading the application. This is accomplished because Android makes developers request permission to all the data and resources that the application will require. The problem is that the average user is unaware of the security and privacy implications behind the permissions screen. Also, the permissions are commonly not read, and instead seen as a click through screen to installing an application. (SOURCE)  

Thankfully, Apple has announced that it will be deprecating the use of the UDID in IOS6 due to the privacy concerns. MORE INFO NEEDED HERE